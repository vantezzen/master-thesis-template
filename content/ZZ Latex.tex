% Basic formatting
\section{Dies ist eine Überschrift}
\subsection{Dies ist eine Unterüberschrift}
\subsubsection{Dies ist eine Unterunterüberschrift}
\paragraph{Dies ist eine Paragraphenüberschrift}
Dies ist ein normaler Text
\textbf{Dies ist fetter Text}
\textit{Dies ist kursiver Text}
\textmono{Dies ist ein monospaced Text}
\medskip
Dieser Text hat ein wenig Abstand zum nächsten Text.
\bigskip
Dieser Text hat mehr Abstand zum nächsten Text.

% Link
\url{https://www.example.com}

% Citation
\textcite{citation_key} described that...
Some text that needs citation \parencite{citation_key}.

% Listing (Code)
% Keep the language always set to PHP - that seems to work best for all C-style languages (TypeScript, C, etc.)
\begin{lstlisting}[frame=htrbl, caption={Caption Here}, label={lst:my listing}, language=php]
class SomeCode extends SomeClass {
    function myFunc() {}
}
\end{lstlisting}

Listing \ref{lst:my listing} zeigt...

% Image
\begin{figure}[h]
\includegraphics[width=0.5\textwidth]{abb/picture.png}
\centering
\caption{Beschreibung \parencite{citation_key}}
\label{fig:my fig}
\end{figure}

Abbildung \ref{fig:my fig} zeigt...
Optional mit Seitenangabe: Abbildung \ref{fig:my fig} [S.\pageref{fig:my fig}] zeigt...


% Tabelle 
\begin{table}[!h]
\begin{center}
    \begin{tabular}{ | l | c | c | c | }
      \hline
      Regel & Parameter 1 & Parameter 2 & Parameter n \\ \hline \hline
      Hauptregel & Kran & Gebäude & ... \\ \hline
    \end{tabular}
\caption{Tabularische Regel nach \textcite{citation_key}}
\label{tab:my table}
\end{center}
\end{table}

Tabelle \ref{tab:my table} zeigt

% Formeln
Einfache Formel:
\begin{equation}
    \mathbf{v}_{forward} = R_{\text{cam}} \cdot (0, 0, -1)
\end{equation}

% Formel mit mehreren Zeilen, zentriert
\begin{equation}
    \begin{aligned}
    & \mathbf{position} = \\
    &    \mathbf{aktuelle position} + \\
    &    \mathbf{v}_{forward} \times \mathbf{touch delta}_{up/down} + \\
    &    \mathbf{v}_{right} \times \mathbf{touch delta}_{left/right}
    \end{aligned}
\end{equation}

% Große Matrix
\begin{align*}
    R = \begin{bmatrix} 
        c_\alpha c_\gamma - s_\alpha s_\beta s_\gamma & -c_\beta s_\alpha & c_\alpha s_\gamma + c_\gamma s_\alpha s_\beta \\
        c_\gamma s_\alpha + c_\alpha s_\beta s_\gamma & c_\alpha c_\beta & s_\alpha s_\gamma - c_\alpha c_\gamma s_\beta \\
        -c_\beta s_\gamma & s_\beta & c_\beta c_\gamma
    \end{bmatrix}
\end{align*}

% Einfache Formel mit Text
\[
x = x_{user} + \textit{Distanz} \cdot \cos(\text{Virtuelle Peilung})
\]
