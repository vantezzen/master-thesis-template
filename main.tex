\documentclass[a4paper,10t,headsepline]{scrartcl}

% Umlaute unter UTF8 nutzen
\usepackage[utf8]{inputenc}

% Variablen
%Variablen welche innerhalb der gesamten Arbeit zur Verfügung stehen sollen
\newcommand{\titleDocument}{Titel der Arbeit}
\newcommand{\subjectDocument}{Masterarbeit}

\newcommand{\authorDocument}{Vorname Nachname}
\newcommand{\erstgutachter}{Prof. Dr. Max Mustermann}
\newcommand{\zweitgutachter}{Prof. Dr. Max Mustermann}


% Grafiken aus PNG Dateien einbinden
\usepackage{graphicx}

% Deutsche Sonderzeichen und Silbentrennung nutzen
\usepackage[ngerman]{babel}

% Eurozeichen einbinden
\usepackage[right]{eurosym}

% Zeichenencoding
\usepackage[T1]{fontenc}

\usepackage{lmodern}

% floatende Bilder ermöglichen
%\usepackage{floatflt}

% mehrseitige Tabellen ermöglichen
\usepackage{longtable}

% Unterstützung für Schriftarten
%\newcommand{\changefont}[3]{ 
%\fontfamily{#1} \fontseries{#2} \fontshape{#3} \selectfont}

% Packet für Seitenrandabständex und Einstellung für Seitenränder
\usepackage{geometry}
\geometry{left=3.5cm, right=2cm, top=2.5cm, bottom=2cm}

% Paket für Boxen im Text
\usepackage{fancybox}

% bricht lange URLs "schön" um
\usepackage{url}

% Paket für Textfarben
\usepackage{color}

% Mathematische Symbole importieren
\usepackage{amssymb}

% auf jeder Seite eine Überschrift (alt, zentriert)
%\pagestyle{headings}

% erzeugt Inhaltsverzeichnis mit Querverweisen zu den Abschnitten (PDF Version)
\usepackage[bookmarksnumbered,pdftitle={\titleDocument},hyperfootnotes=false]{hyperref}
%\hypersetup{colorlinks, citecolor=red, linkcolor=blue, urlcolor=black}
%\hypersetup{colorlinks, citecolor=black, linkcolor= black, urlcolor=black}

% neue Kopfzeilen mit fancypaket
\usepackage{fancyhdr} %Paket laden
\pagestyle{fancy} %eigener Seitenstil
\fancyhf{} %alle Kopf- und Fußzeilenfelder bereinigen
\fancyhead[L]{\nouppercase{\leftmark}} %Kopfzeile links
\fancyhead[C]{} %zentrierte Kopfzeile
\fancyhead[R]{\thepage} %Kopfzeile rechts
\renewcommand{\headrulewidth}{0.4pt} %obere Trennlinie
%\fancyfoot[C]{\thepage} %Seitennummer
%\renewcommand{\footrulewidth}{0.4pt} %untere Trennlinie

% FloatBarriers
\usepackage{placeins}

% für Tabellen
\usepackage{array}
\usepackage{makecell}
\renewcommand\theadalign{bc}
\renewcommand\theadfont{\bfseries}
\renewcommand\theadgape{\Gape[4pt]}
\renewcommand\cellgape{\Gape[4pt]}

% Festlegung Art der Zitierung
\usepackage[
    backend=biber,
    style=authoryear,
    maxcitenames=1,
    sorting=nyc
]{biblatex}
\DefineBibliographyStrings{ngerman}{
   andothers = {{et\,al\adddot}},
}
% https://tex.stackexchange.com/a/389164/278358
\DeclareSortingScheme{nyc}{
  \sort{
    \field{presort}
  }
  \sort[final]{
    \field{sortkey}
  }
  \sort{
    \field{sortname}
    \field{author}
    \field{editor}
    \field{translator}
    \field{sorttitle}
    \field{title}
  }
  \sort{
    \field{sortyear}
    \field{year}
  }
  \sort{\citeorder}
}
\addbibresource{bibliography/main.bib}
\addbibresource{bibliography/Bilder.bib}

\usepackage{csquotes}
\usepackage{microtype}

% Festlegung Art der Zitierung - Havardmethode: Abkuerzung Autor + Jahr
%\bibliographystyle{alphadin}
%\newcommand*{\quelle}{%
%  \footnotesize Quelle:
%}

% Schaltet den zusätzlichen Zwischenraum ab, den LaTeX normalerweise nach einem Satzzeichen einfügt.
%\frenchspacing

% Paket für Zeilenabstand
\usepackage{setspace}

% für Bildbezeichner
\usepackage{capt-of}

% für Stichwortverzeichnis
\usepackage{makeidx}

%Konfiguriere das Inhaltsverzeichnis
\usepackage{tocbasic}
\setcounter{secnumdepth}{5}
\DeclareTOCStyleEntries[
  raggedentrytext,
  numwidth=0pt,
  numsep=1ex,
  dynnumwidth,
]{tocline}{chapter,section,subsection,subsubsection,paragraph,subparagraph}
\DeclareTOCStyleEntries[
  linefill=\TOCLineLeaderFill,
]{tocline}{section,subsection,subsubsection,paragraph,subparagraph}


% für Listings
\usepackage{listings}
\usepackage{xcolor}

\definecolor{codegreen}{rgb}{0,0.6,0}
\definecolor{codegray}{rgb}{0.5,0.5,0.5}
\definecolor{codepurple}{rgb}{0.58,0,0.82}
\definecolor{backcolour}{rgb}{0.95,0.95,0.92}

\lstdefinestyle{mystyle}{
    backgroundcolor=\color{backcolour},   
    commentstyle=\color{codegreen},
    keywordstyle=\color{magenta},
    numberstyle=\tiny\color{codegray},
    stringstyle=\color{codepurple},
    basicstyle=\ttfamily\footnotesize,
    breakatwhitespace=false,         
    breaklines=true,                 
    captionpos=b,                    
    keepspaces=true,                 
    numbers=left,                    
    numbersep=5pt,                  
    showspaces=false,                
    showstringspaces=false,
    showtabs=false,                  
    tabsize=2
}
\lstset{
    style=mystyle,
    mathescape=false
}

\usepackage{xcolor}
\definecolor{light-gray}{gray}{0.95}
\newcommand{\textmono}[1]{\colorbox{light-gray}{\texttt{#1}}}

% Indexerstellung
\makeindex

% Abkürzungsverzeichnis
\usepackage[german]{nomencl}
\let\abbrev\nomenclature

% Abkürzungsverzeichnis LiveTex Version
% Titel des Abkürzungsverzeichnisses
\renewcommand{\nomname}{Abkürzungsverzeichnis}
% Abstand zwischen Abkürzung und Erläuterung
\setlength{\nomlabelwidth}{.25\textwidth}
% Zwischenraum zwischen Abkürzung und Erläuterung mit Punkten
\renewcommand{\nomlabel}[1]{#1 \dotfill}
% Variation des Abstandes der einzelnen Abkürzungen zu einander
\setlength{\nomitemsep}{-\parsep}
% Index mit Abkürzungen erzeugen
\makenomenclature
%\makeglossary

\begin{document}
\input{settings/trennung}

% Schriftart Helvetica verwenden
%\usepackage{helvet}
%\renewcommand\familydefault{\sfdefault}

% Leere Seite am Anfang
%\thispagestyle{empty} % erzeugt Seite ohne Kopf- / Fusszeile
%\mbox{}
%\newpage
% Titelseite %
\thispagestyle{empty}


\begin{figure}[t]
 \centering
 \includegraphics[width=0.6\textwidth]{abb/htw-logo.png}
\end{figure}


\begin{verbatim}


\end{verbatim}

\begin{center}
\doublespacing
\textbf{\LARGE{\titleDocument}}\\
\singlespacing
\begin{verbatim}

\end{verbatim}
\textbf{{~\subjectDocument}}

Angewandte Informatik \\
Fachbereich 4
\end{center}
\begin{verbatim}

\end{verbatim}
\begin{center}

\end{center}
\begin{verbatim}

\end{verbatim}
\begin{flushleft}
\begin{tabular}{llll}
\textbf{Vorgelegt von:} & & \authorDocument & \\
%& & \\
\textbf{Darum:} & & Berlin, \today &\\
& & \\
% \textbf{Betreuer:} & & N/A

\textbf{Erstgutachter:} & & \erstgutachter &\\
\textbf{Zweitgutachter:} & & \zweitgutachter &\\


\end{tabular}
\end{flushleft}


% römische Numerierung
\pagenumbering{roman}

% 1.5 facher Zeilenabstand
\onehalfspacing

\newpage

% Einleitung / Abstract
\thispagestyle{empty}
\section*{Abstract}\label{section:Abstract}



% einfacher Zeilenabstand
\singlespacing

\newpage
% Seitenzählung bei Inhaltsverzeichnis beginnen
\setcounter{page}{1}

% Inhaltsverzeichnis anzeigen
\thispagestyle{empty}
\tableofcontents

\newpage
% das Abbildungsverzeichnis
\listoffigures
% Abbildungsverzeichnis soll im Inhaltsverzeichnis auftauchen
\addcontentsline{toc}{section}{Abbildungsverzeichnis}

% das Tabellenverzeichnis
\newpage
 \fancyhead[L]{Abbildungsverzeichnis / Abkürzungsverzeichnis} %Kopfzeile links
% Tabellenverzeichnis endgültig anzeigen
\listoftables
% Tabellenverzeichnis soll im Inhaltsverzeichnis auftauchen
\addcontentsline{toc}{section}{Tabellenverzeichnis}

%% WORKAROUND für Listings
%\makeatletter% --> De-TeX-FAQ
%\renewcommand*{\lstlistoflistings}{%
%  \begingroup
%    \if@twocolumn
%      \@restonecoltrue\onecolumn
%    \else
%      \@restonecolfalse
%    \fi
%    \lol@heading
%    \setlength{\parskip}{\z@}%
%    \setlength{\parindent}{\z@}%
%    \setlength{\parfillskip}{\z@ \@plus 1fil}%
%    \@starttoc{lol}%
%    \if@restonecol\twocolumn\fi
%  \endgroup
%}
%\makeatother% --> \makeatletter
% das Listingverzeichnis
\newpage
\fancyhead[L]{Listingverzeichnis} %Kopfzeile links
\renewcommand{\lstlistlistingname}{Listingverzeichnis}
\lstlistoflistings
% Listingverzeichnis soll im Inhaltsverzeichnis auftauchen
\addcontentsline{toc}{section}{Listingverzeichnis}
%%%%

% das Abkürzungsverzeichnis
\newpage
% das Abkürzungsverzeichnis ausgeben
\fancyhead[L]{Abkürzungsverzeichnis} %Kopfzeile links

\nomenclature{BIM}{Building Information Modeling}
\nomenclature{CAD}{Computer-aided design}
\nomenclature{HAR}{Handheld Augmented Reality}
\nomenclature{HCI}{Human-Computer Interaction}

\printnomenclature[3cm]
% Abkürzungsverzeichnis soll im Inhaltsverzeichnis auftauchen
\addcontentsline{toc}{section}{Abkürzungsverzeichnis}


%%%%%%% EINLEITUNG %%%%%%%%%%%%
\newpage
\fancyhead[L]{\nouppercase{\leftmark}} %Kopfzeile links

% 1,5 facher Zeilenabstand
\onehalfspacing

% arabische Seitennummerierung ab hier
\pagenumbering{arabic}

% Alternative Einbindung des Abstract in Kapitel "0" falls gewünscht
%\setcounter{section}{-1}
%\setcounter{page}{0}

% Option: Einbindung abstract
%\input{0_abstract}
%\newpage

% einzelne Kapitel werden hier eingebunden
\section{Einleitung}\label{section:Einleitung}
\newpage

\section{Grundlagen}\label{section:Grundlagen}

\newpage

\section{Verwandte Arbeiten}\label{section:Verwandte Arbeiten}

\newpage

\section{Konzeptualisierung}\label{section:Konzeptualisierung}
\newpage

\section{Konzeptualisierung Systems}\label{section:Konzeptualisierung Systems}
\newpage

\section{Implementierung}\label{section:Implementierung}
\newpage

\section{Usertesting des Systems}\label{section:Usertesting des Systems}
\newpage

\section{Diskussion}\label{section:Diskussion}
\newpage

\section{Fazit}\label{section:Fazit}

%\input{beispiel}

% einfacher Zeilenabstand
\singlespacing
% Literaturliste soll im Inhaltsverzeichnis auftauchen
\newpage
\phantomsection
\addcontentsline{toc}{section}{Literaturverzeichnis}
% Literaturverzeichnis anzeigen
\renewcommand\refname{Literaturverzeichnis}
\printbibliography

%% Index soll Stichwortverzeichnis heissen
% \newpage
% % Stichwortverzeichnis soll im Inhaltsverzeichnis auftauchen
% \addcontentsline{toc}{section}{Stichwortverzeichnis}
% \renewcommand{\indexname}{Stichwortverzeichnis}
% % Stichwortverzeichnis endgültig anzeigen
% \printindex

\onehalfspacing

% Anhang
\newpage
\phantomsection
\addcontentsline{toc}{section}{Anhang}
\fancyhead[L]{Anhang} %Kopfzeile links
\input{anhang/anhang}

% Eidesstattliche Erklärung
\newpage
\phantomsection
\addcontentsline{toc}{section}{Eidesstattliche Erklärung}
\input{envelope/erklaerung}

% leere Abschlussseite
\newpage
\thispagestyle{empty} % erzeugt Seite ohne Kopf- / Fusszeile
\mbox{}

\end{document}
